% Reference Card for slime
%**start of header
\newcount\columnsperpage
\newcount\letterpaper

% This file can be printed with 1, 2, or 3 columns per page (see below).
% Specify how many you want here.

\columnsperpage=3

% Set letterpapaer to 0 for A4 paper, 1 for letter (US) paper.  Useful
% only when columnsperpage is 2 or 3.

\letterpaper=0

% Nothing else needs to be changed below this line.
% Copyright (C) 1987, 1993, 1996, 1997, 2002, 2003, 2004,
%   2005 Free Software Foundation, Inc.
% Copyright (C) 2009 Rob Myers rob@robmyers.org

% GNU Emacs is free software; you can redistribute it and/or modify
% it under the terms of the GNU General Public License as published by
% the Free Software Foundation; either version 2, or (at your option)
% any later version.

% GNU Emacs is distributed in the hope that it will be useful,
% but WITHOUT ANY WARRANTY; without even the implied warranty of
% MERCHANTABILITY or FITNESS FOR A PARTICULAR PURPOSE.  See the
% GNU General Public License for more details.

% You should have received a copy of the GNU General Public License
% along with GNU Emacs; see the file COPYING.  If not, write to
% the Free Software Foundation, Inc., 51 Franklin Street, Fifth Floor,
% Boston, MA 02110-1301, USA.

% This file is intended to be processed by plain TeX (TeX82).
%
% The final reference card has six columns, three on each side.
% This file can be used to produce it in any of three ways:
% 1 column per page
%    produces six separate pages, each of which needs to be reduced to 80%.
%    This gives the best resolution.
% 2 columns per page
%    produces three already-reduced pages.
%    You will still need to cut and paste.
% 3 columns per page
%    produces two pages which must be printed sideways to make a
%    ready-to-use 8.5 x 11 inch reference card.
%    For this you need a dvi device driver that can print sideways.
% Which mode to use is controlled by setting \columnsperpage above.
%
% To compile and print this document:
% tex refcard.tex
% dvips -t landscape refcard.dvi
%
% Author:
%  Stephen Gildea
%  Internet: gildea@stop.mail-abuse.org
%
% Thanks to Paul Rubin, Bob Chassell, Len Tower, and Richard Mlynarik
% for their many good ideas.

\def\versionnumber{0.1}
\def\year{2009}

\def\shortcopyrightnotice{}

\def\copyrightnotice{}

% make \bye not \outer so that the \def\bye in the \else clause below
% can be scanned without complaint.
\def\bye{\par\vfill\supereject\end}

\newdimen\intercolumnskip	%horizontal space between columns
\newbox\columna			%boxes to hold columns already built
\newbox\columnb

\def\ncolumns{\the\columnsperpage}

\message{[\ncolumns\space
  column\if 1\ncolumns\else s\fi\space per page]}

\def\scaledmag#1{ scaled \magstep #1}

% This multi-way format was designed by Stephen Gildea October 1986.
% Note that the 1-column format is fontfamily-independent.
\if 1\ncolumns			%one-column format uses normal size
  \hsize 4in
  \vsize 10in
  \voffset -.7in
  \font\titlefont=\fontname\tenbf \scaledmag3
  \font\headingfont=\fontname\tenbf \scaledmag2
  \font\smallfont=\fontname\sevenrm
  \font\smallsy=\fontname\sevensy

  \footline{\hss\folio}
  \def\makefootline{\baselineskip10pt\hsize6.5in\line{\the\footline}}
\else				%2 or 3 columns uses prereduced size
  \hsize 3.2in
  \if 1\the\letterpaper
     \vsize 7.95in
  \else
     \vsize 7.65in
  \fi
  %\hoffset -.75in % If the output is too far to the right, uncomment this
  \voffset -.745in
  \font\titlefont=cmbx10 \scaledmag2
  \font\headingfont=cmbx10 \scaledmag1
  \font\smallfont=cmr6
  \font\smallsy=cmsy6
  \font\eightrm=cmr8
  \font\eightbf=cmbx8
  \font\eightit=cmti8
  \font\eighttt=cmtt8
  \font\eightmi=cmmi8
  \font\eightsy=cmsy8
  \textfont0=\eightrm
  \textfont1=\eightmi
  \textfont2=\eightsy
  \def\rm{\eightrm}
  \def\bf{\eightbf}
  \def\it{\eightit}
  \def\tt{\eighttt}
  \if 1\the\letterpaper
     \normalbaselineskip=.8\normalbaselineskip
  \else
     \normalbaselineskip=.7\normalbaselineskip
  \fi
  \normallineskip=.8\normallineskip
  \normallineskiplimit=.8\normallineskiplimit
  \normalbaselines\rm		%make definitions take effect

  \if 2\ncolumns
    \let\maxcolumn=b
    \footline{\hss\rm\folio\hss}
    \def\makefootline{\vskip 2in \hsize=6.86in\line{\the\footline}}
  \else \if 3\ncolumns
    \let\maxcolumn=c
    \nopagenumbers
  \else
    \errhelp{You must set \columnsperpage equal to 1, 2, or 3.}
    \errmessage{Illegal number of columns per page}
  \fi\fi

  \intercolumnskip=.46in
  \def\abc{a}
  \output={%			%see The TeXbook page 257
      % This next line is useful when designing the layout.
      %\immediate\write16{Column \folio\abc\space starts with \firstmark}
      \if \maxcolumn\abc \multicolumnformat \global\def\abc{a}
      \else\if a\abc
	\global\setbox\columna\columnbox \global\def\abc{b}
        %% in case we never use \columnb (two-column mode)
        \global\setbox\columnb\hbox to -\intercolumnskip{}
      \else
	\global\setbox\columnb\columnbox \global\def\abc{c}\fi\fi}
  \def\multicolumnformat{\shipout\vbox{\makeheadline
      \hbox{\box\columna\hskip\intercolumnskip
        \box\columnb\hskip\intercolumnskip\columnbox}
      \makefootline}\advancepageno}
  \def\columnbox{\leftline{\pagebody}}

  \def\bye{\par\vfill\supereject
    \if a\abc \else\null\vfill\eject\fi
    \if a\abc \else\null\vfill\eject\fi
    \end}
\fi

% we won't be using math mode much, so redefine some of the characters
% we might want to talk about
\catcode`\^=12
\catcode`\_=12

\chardef\\=`\\
\chardef\{=`\{
\chardef\}=`\}

\hyphenation{mini-buf-fer}

\parindent 0pt
\parskip 1ex plus .5ex minus .5ex

\def\small{\smallfont\textfont2=\smallsy\baselineskip=.8\baselineskip}

% newcolumn - force a new column.  Use sparingly, probably only for
% the first column of a page, which should have a title anyway.
\outer\def\newcolumn{\vfill\eject}

% title - page title.  Argument is title text.
\outer\def\title#1{{\titlefont\centerline{#1}}\vskip 1ex plus .5ex}

% section - new major section.  Argument is section name.
\outer\def\section#1{\par\filbreak
  \vskip 3ex plus 2ex minus 2ex {\headingfont #1}\mark{#1}%
  \vskip 2ex plus 1ex minus 1.5ex}

\newdimen\keyindent

% beginindentedkeys...endindentedkeys - key definitions will be
% indented, but running text, typically used as headings to group
% definitions, will not.
\def\beginindentedkeys{\keyindent=1em}
\def\endindentedkeys{\keyindent=0em}
\endindentedkeys

% paralign - begin paragraph containing an alignment.
% If an \halign is entered while in vertical mode, a parskip is never
% inserted.  Using \paralign instead of \halign solves this problem.
\def\paralign{\vskip\parskip\halign}

% \<...> - surrounds a variable name in a code example
\def\<#1>{{\it #1\/}}

% kbd - argument is characters typed literally.  Like the Texinfo command.
\def\kbd#1{{\tt#1}\null}	%\null so not an abbrev even if period follows

% beginexample...endexample - surrounds literal text, such a code example.
% typeset in a typewriter font with line breaks preserved
\def\beginexample{\par\leavevmode\begingroup
  \obeylines\obeyspaces\parskip0pt\tt}
{\obeyspaces\global\let =\ }
\def\endexample{\endgroup}

% key - definition of a key.
% \key{description of key}{key-name}
% prints the description left-justified, and the key-name in a \kbd
% form near the right margin.
\def\key#1#2{\leavevmode\hbox to \hsize{\vtop
  {\hsize=.75\hsize\rightskip=1em
  \hskip\keyindent\relax#1}\kbd{#2}\hfil}}

\newbox\metaxbox
\setbox\metaxbox\hbox{\kbd{M-x }}
\newdimen\metaxwidth
\metaxwidth=\wd\metaxbox

% metax - definition of a M-x command.
% \metax{description of command}{M-x command-name}
% Tries to justify the beginning of the command name at the same place
% as \key starts the key name.  (The "M-x " sticks out to the left.)
\def\metax#1#2{\leavevmode\hbox to \hsize{\hbox to .75\hsize
  {\hskip\keyindent\relax#1\hfil}%
  \hskip -\metaxwidth minus 1fil
  \kbd{#2}\hfil}}

% threecol - like "key" but with two key names.
% for example, one for doing the action backward, and one for forward.
\def\threecol#1#2#3{\hskip\keyindent\relax#1\hfil&\kbd{#2}\hfil\quad
  &\kbd{#3}\hfil\quad\cr}

%**end of header


% \key{}{}
% {\bf bold}
\newcolumn
\title{Minara Reference Card}

\centerline{(for version 0.3)}

\section{Starting Minara}

To start Minara from the command line, type: \kbd{minara}

\section{Keymaps}

\key{return to global keymap}{C-g}

\section{Development}

\key{reload user's .minara file}{d d}
\key{reload libary and tool files}{d I}
\key{reload tool files}{d t}

\section{Drawing}

\key{circle}{t c}
\key{simple pen}{t p}
\key{polyline tool}{t P}
\key{rectangle}{t r}
\key{square}{t r}
\key{square}{t s}
\key{star}{t S}

\section{Files}

\key{edit user's .minara file}{x d}
\key{edit current file text}{x e}
\key{reload current file}{x r}

\section{Colour}

\key{edit current rgb colour}{c r}

\section{Colour Keymap}

\key{set colour and return}{q}
\key{set current component to red}{r}
\key{set current component to green}{g}
\key{set current component to blue}{b}
\key{increase current component by 0.01}{+}
\key{increase current component by 0.1}{=}
\key{reduce current component by 0.1}{-}
\key{reduce current component by 0.01}{_}
\key{set current component to 0.1 (or 0.x for other numbers)}{1}
\key{apropos for all symbols in package}{C-c C-d p}
\key{look up symbol at point in hyperspec}{C-c C-d h}
\key{look up format character in hyperspec}{C-c C-d ~}

\section{Undo}

\key{undo}{z}
\key{redo}{Z}

\section{View}

\key{zoom in}{i}
\key{zoom out}{I}
\key{default zoom}{alt-i}
\key{pan}{p}
\key{default pan}{P}
\key{Panic!}{alt-P}

\section{Selection}

\key{select}{t s}
\key{select none}{s SPACE}
\key{clear selection}{s c}
\key{copy}{C-c}
\key{delete}{C-d}
\key{cut}{C-x}
\key{paste}{C-v}
\key{list callers of function}{C-c <}
\key{list callees of function}{C-c >}

\section{Window}

\key{close}{x c}
\key{save}{x s}
\key{reload}{x r}
\key{edit externally}{x e}

%\metax{do something}{M-x do something}


%\newcolumn
%\title{Minara Reference Card}
%\copyrightnotice

\bye

% Local variables:
% compile-command: "tex refcard"
% End:
